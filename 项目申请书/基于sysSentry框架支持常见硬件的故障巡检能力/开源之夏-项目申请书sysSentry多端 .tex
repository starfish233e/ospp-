% Options for packages loaded elsewhere
\PassOptionsToPackage{unicode}{hyperref}
\PassOptionsToPackage{hyphens}{url}
%
\documentclass[
]{article}
\usepackage{amsmath,amssymb}
\usepackage{iftex}
\ifPDFTeX
  \usepackage[T1]{fontenc}
  \usepackage[utf8]{inputenc}
  \usepackage{textcomp} % provide euro and other symbols
\else % if luatex or xetex
  \usepackage{unicode-math} % this also loads fontspec
  \defaultfontfeatures{Scale=MatchLowercase}
  \defaultfontfeatures[\rmfamily]{Ligatures=TeX,Scale=1}
\fi
\usepackage{lmodern}
\ifPDFTeX\else
  % xetex/luatex font selection
\fi
% Use upquote if available, for straight quotes in verbatim environments
\IfFileExists{upquote.sty}{\usepackage{upquote}}{}
\IfFileExists{microtype.sty}{% use microtype if available
  \usepackage[]{microtype}
  \UseMicrotypeSet[protrusion]{basicmath} % disable protrusion for tt fonts
}{}
\makeatletter
\@ifundefined{KOMAClassName}{% if non-KOMA class
  \IfFileExists{parskip.sty}{%
    \usepackage{parskip}
  }{% else
    \setlength{\parindent}{0pt}
    \setlength{\parskip}{6pt plus 2pt minus 1pt}}
}{% if KOMA class
  \KOMAoptions{parskip=half}}
\makeatother
\usepackage{xcolor}
\usepackage{graphicx}
\makeatletter
\def\maxwidth{\ifdim\Gin@nat@width>\linewidth\linewidth\else\Gin@nat@width\fi}
\def\maxheight{\ifdim\Gin@nat@height>\textheight\textheight\else\Gin@nat@height\fi}
\makeatother
% Scale images if necessary, so that they will not overflow the page
% margins by default, and it is still possible to overwrite the defaults
% using explicit options in \includegraphics[width, height, ...]{}
\setkeys{Gin}{width=\maxwidth,height=\maxheight,keepaspectratio}
% Set default figure placement to htbp
\makeatletter
\def\fps@figure{htbp}
\makeatother
\setlength{\emergencystretch}{3em} % prevent overfull lines
\providecommand{\tightlist}{%
  \setlength{\itemsep}{0pt}\setlength{\parskip}{0pt}}
\setcounter{secnumdepth}{-\maxdimen} % remove section numbering
\ifLuaTeX
  \usepackage{selnolig}  % disable illegal ligatures
\fi
\usepackage{bookmark}
\IfFileExists{xurl.sty}{\usepackage{xurl}}{} % add URL line breaks if available
\urlstyle{same}
\hypersetup{
  hidelinks,
  pdfcreator={LaTeX via pandoc}}

\author{}
\date{}

\begin{document}

\section{开源之夏-项目申请书}\label{ux5f00ux6e90ux4e4bux590f-ux9879ux76eeux7533ux8bf7ux4e66}

\subsection{项目名称:基于sysSentry框架支持常见硬件的故障巡检能力}\label{ux9879ux76eeux540dux79f0ux57faux4e8esyssentryux6846ux67b6ux652fux6301ux5e38ux89c1ux786cux4ef6ux7684ux6545ux969cux5de1ux68c0ux80fdux529b}

\textbf{项目编号:24b970341}

\paragraph{\texorpdfstring{项目导师邮箱:\href{mailto:zhangnan134@huawei.com}{\nolinkurl{zhangnan134@huawei.com}}
}{项目导师邮箱:zhangnan134@huawei.com }}\label{ux9879ux76eeux5bfcux5e08ux90aeux7bb1zhangnan134huaweicom}

\subsection{\texorpdfstring{申请人:
}{申请人: }}\label{ux7533ux8bf7ux4eba}

\paragraph{\texorpdfstring{个人邮件:\href{mailto:2789954149@qq.com}{\nolinkurl{2789954149@qq.com}}}{个人邮件:2789954149@qq.com}}\label{ux4e2aux4ebaux90aeux4ef62789954149qqcom}

\paragraph{\texorpdfstring{个人主页:\url{https://gitee.com/xu-zihanhan}}{个人主页:https://gitee.com/xu-zihanhan}}\label{ux4e2aux4ebaux4e3bux9875httpsgiteecomxu-zihanhan}

\#\#\#

\subsection{基于sysSentry框架支持常见硬件的故障巡检能力}\label{ux57faux4e8esyssentryux6846ux67b6ux652fux6301ux5e38ux89c1ux786cux4ef6ux7684ux6545ux969cux5de1ux68c0ux80fdux529b}

\subsubsection{项目背景}\label{ux9879ux76eeux80ccux666f}

\begin{enumerate}
\def\labelenumi{\arabic{enumi}.}
\item
  \textbf{实时监控需求}:sysSentry是一个统一故障巡检框架,支持针对OS内的各类硬件进行故障压测和巡检,并提供统一的结果以及告警通知能力。运维平台可以通过调用sysSentry提供的统一接口进行OS上的统一巡检,提前发现故障并进行处理,提升系统可靠性。随着IT基础设施的日益复杂,对硬件设备的实时监控和故障预警成为关键需求。
\item
  \textbf{Sentry框架优势}:当前在sysSentry中已经支持CPU、内存、NPU等硬件的巡检,该项目主要是要在sysSentry中支持新的硬件巡检能力,通过提供插件的方式,支持DPU、FC、GPU、IB、NIC、RAID、SSD、密码卡等,支持的硬件类型以及型号参考兼容性列表:{[}\url{https://www.openeuler.org/zh/compatibility}
  作为成熟的错误监控和上报平台,支持多语言、多平台,并已经在大量组织中成功应用,提供了丰富的API支持和社区支持。
\item
  \textbf{故障巡检目标}:对于各类硬件的巡检能力,可以主要可以借助BMC以及各类硬件提供的巡检工具进行插件功能的开发。集成sysSentry框架,本项目将实现对硬件设备的实时故障巡检,及时发现并预警潜在问题,提高系统的稳定性和可靠性。
\end{enumerate}

\subsubsection{项目目标}\label{ux9879ux76eeux76eeux6807}

\textbf{1、按照sysSentry框架对接规范开发对应的插件;}\\
\textbf{2、需要提出对应的测试用例;}\\
\textbf{3、通过仓库的门禁检查;(做好测试,安全性分析)}

\subsubsection{技术要求}\label{ux6280ux672fux8981ux6c42}

\begin{enumerate}
\def\labelenumi{\arabic{enumi}.}
\item
  悉Python语言开发
\item
  对于Linux下常见的硬件故障有一定的了解或者一定的学习能力
\end{enumerate}

\subsubsection{技术栈对应}\label{ux6280ux672fux6808ux5bf9ux5e94}

\begin{enumerate}
\def\labelenumi{\arabic{enumi}.}
\item
  本人熟悉 sysSentry 框架,对嵌入式 Linux 工控系统有一定了解。
\item
  容器相关知识 Docker 部署过 本地环境,了解前端 Figma 设计
\item
  操作系统知识 了解 openEuler 操作系统,贡献过博客阅读量:22074

  参与博客、项目小组平台 new\_energy\_coder\_club 开源经历贡献
\item
  编程语言标签:Python、Shell

  \subsubsection{- 项目经历}\label{--ux9879ux76eeux7ecfux5386}

  ARH502-CN 边缘计算网关 部署-容器应用 MyBatis/ModelArts 模型

  2024 全国大学生物联网设计竞赛(华为杯 IOT)高校赛道-工控赛道选手

  \subsubsection{- 知识储备}\label{--ux77e5ux8bc6ux50a8ux5907}

  了解 Yocto 框架与嵌入式 Linux 系统【最近在学习编译】,有 Redis 经验

  会使用 soildworks(工控硬件设备设计, 硬件光通讯系统设计经验)
\end{enumerate}

\subsubsection{项目安排}\label{ux9879ux76eeux5b89ux6392}

\paragraph{第一阶段:环境部署,硬件兼容(7月1日至7月15日)}\label{ux7b2cux4e00ux9636ux6bb5ux73afux5883ux90e8ux7f72ux786cux4ef6ux517cux5bb97ux67081ux65e5ux81f37ux670815ux65e5uxff09}

\textbf{时间安排}:项目启动至第二周结束

\textbf{主要任务}:

\begin{enumerate}
\def\labelenumi{\arabic{enumi}.}
\item
  下载部署sysSentry的环境
\item
  \includegraphics{/Users/imac/Library/Application Support/typora-user-images/image-20240604154621394.png}
\item
  在本地运行部署sysSentry
\end{enumerate}

\paragraph{第二阶段:接入硬件层API,MQTT接入(7月16日至8月5日)}\label{ux7b2cux4e8cux9636ux6bb5ux63a5ux5165ux786cux4ef6ux5c42apimqttux63a5ux51657ux670816ux65e5ux81f38ux67085ux65e5uxff09}

\textbf{时间安排}:第三周至第五周结束

\textbf{主要任务}:

\begin{enumerate}
\def\labelenumi{\arabic{enumi}.}
\item
  封装硬件API来集成到巡检任务中
\item
  \includegraphics{/Users/imac/Library/Application Support/typora-user-images/image-20240604155018063.png}
\item
  我们本地有一辆小车,上位机通过MQTT与我们的主机AR502H/CN通讯,我们想要将通讯接口封装成API
\end{enumerate}

\paragraph{第三阶段:多端架构测试优化与扩展(8月6日至8月15日)}\label{ux7b2cux4e09ux9636ux6bb5ux591aux7aefux67b6ux6784ux6d4bux8bd5ux4f18ux5316ux4e0eux6269ux5c558ux67086ux65e5ux81f38ux670815ux65e5uxff09}

\textbf{时间安排}:第六周至第八周结束

\textbf{主要任务}:

\begin{enumerate}
\def\labelenumi{\arabic{enumi}.}
\item
  在硬件管理巡检程序中集成sysSentry,通过yocto编译系统镜像运行本地openEuler
  操作系统。
\item
  编写调用API的脚本程序,连接sysSentry,产出DEMO,贡献镜像到openEuler
  社区。
\end{enumerate}

\paragraph{第四阶段:文档完善与用户支持(8月16日至9月26日)}\label{ux7b2cux56dbux9636ux6bb5ux6587ux6863ux5b8cux5584ux4e0eux7528ux6237ux652fux63018ux670816ux65e5ux81f39ux670826ux65e5uxff09}

\textbf{时间安排}:第九周至项目结束

\textbf{主要任务}:

\begin{enumerate}
\def\labelenumi{\arabic{enumi}.}
\item
  撰写详尽的文档,撰写详细的测试报告,包括测试结果、问题修复情况、测试覆盖率等关键信息,为项目导
  师及社区决策提供依据\textbf{(}按照\textbf{openEuler}
  社区\textbf{G11}文档标准撰写,同期将测试进度更新在仓
  库\textbf{readme)}
\item
  检查代码质量,通过社区门禁
\item
  编写用户手册,详细介绍如何使用\textbf{sysSentry}工具进行单元测试,帮助用户快速上手并充分
  利用其功能。

  \textbf{3.}在提交\textbf{PR}之前,持续维护更新在社区论坛保持活跃,体验\textbf{sysSentry}工具的功能和用户体
  验,满足用户需求。
\end{enumerate}

\paragraph{项目里程碑}\label{ux9879ux76eeux91ccux7a0bux7891}

\begin{itemize}
\item
  第一阶段结束:测试框架成功集成到sysSentry工具中,初始测试验证通过。
\item
  第二阶段结束:全面单元测试开发完成,测试报告提交。
\item
  第三阶段结束:测试框架和脚本优化完成,扩展测试范围实现。
\item
  项目结束:Readme文档完善,用户支持流程建立,项目总结报告提交。
\end{itemize}

\includegraphics{/Users/imac/Library/Application Support/typora-user-images/image-20240604160919460.png}

\subsection{在项目后期,如果有时间将会拉出本项目在多端部署Android\textbackslash iOS\textbackslash Harmony
OS \textbackslash openEuler 多端操作系统的统一 NT 架构
APP横向对比,并对openEuler
进行优化}\label{ux5728ux9879ux76eeux540eux671fux5982ux679cux6709ux65f6ux95f4ux5c06ux4f1aux62c9ux51faux672cux9879ux76eeux5728ux591aux7aefux90e8ux7f72androidiosharmony-os-openeuler-ux591aux7aefux64cdux4f5cux7cfbux7edfux7684ux7edfux4e00-nt-ux67b6ux6784-appux6a2aux5411ux5bf9ux6bd4ux5e76ux5bf9openeuler-ux8fdbux884cux4f18ux5316}

\paragraph{在遇到问题时,将及时向openEuler
社区论坛和社群反馈,积极寻求帮助和解决问题的方法。}\label{ux5728ux9047ux5230ux95eeux9898ux65f6ux5c06ux53caux65f6ux5411openeuler-ux793eux533aux8bbaux575bux548cux793eux7fa4ux53cdux9988ux79efux6781ux5bfbux6c42ux5e2eux52a9ux548cux89e3ux51b3ux95eeux9898ux7684ux65b9ux6cd5}

\paragraph{\texorpdfstring{仓库链接:\url{https://gitee.com/openeuler/sysSentry}}{仓库链接:https://gitee.com/openeuler/sysSentry}}\label{ux4ed3ux5e93ux94feux63a5httpsgiteecomopeneulersyssentry}

\paragraph{\texorpdfstring{文档地址:\url{https://www.openeuler.org/zh/compatibility/}}{文档地址:https://www.openeuler.org/zh/compatibility/}}\label{ux6587ux6863ux5730ux5740httpswwwopeneulerorgzhcompatibility}

\textbf{GitHub/Gitee ID:}

\textbf{项目编号}

\end{document}
